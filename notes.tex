\documentclass[11pt,letterpaper]{article}
\usepackage[pdftex]{graphicx}
\usepackage{amsmath}
\usepackage{amssymb}
\usepackage{fullpage}
\usepackage[utf8]{inputenc}  % char encoding
\pagestyle{plain}   % do page numbering ('empty' turns off)
\usepackage{listings}
\lstloadlanguages{R}
\lstset{language=R,
basicstyle=\footnotesize,
tabsize=2,
columns=fullflexible}
\usepackage[colorlinks=true,linkcolor=blue,urlcolor=blue]{hyperref}

\DeclareMathOperator*{\argmax}{arg\,max}
\DeclareMathOperator*{\argmin}{arg\,min}
\DeclareMathOperator*{\logit}{logit}

\title{Survival Time Analysis with Mixtures of Accelerated Failure Time (AFT) Model Experts}
\author{Daniel Maturana and Abel Valdebenito}

\begin{document}
\maketitle

\section{Accelerated Failure Time Models}

A parametric model for survival time data.

Survival Function for AFT
\begin{align*}
    S(t|x) = \psi \left(  \frac{ \log t - x'\beta}{\sigma}\right)
\end{align*}

For log-normal AFT,
\begin{align*}
    \psi(\cdot) = 1 - \Phi(\cdot)
\end{align*}
So
\begin{align*}
    F(t) &= 1 - S(t) \\
         &= \Phi\left( \frac{\log t - x'\beta}{\sigma} \right) \\
    f(t) &= F'(t) \\
         &= \phi\left( \frac{\log t - x'\beta}{\sigma} \right) \frac{1}{ \sigma t } \\
         &= \frac{1}{\sqrt{2 \pi}} \exp\left( \frac{ -(\log t - x'\beta)^2 }{ 2\sigma^2} \right) \frac{1}{\sigma t} \\
         &= \frac{1}{\sqrt{2 \pi \sigma^2}} \exp\left( \frac{ -(\log t - x'\beta)^2 }{ 2\sigma^2} \right) \frac{1}{t} \\
         &= \mathcal{N}( \log t | x'\beta, \sigma^2 ) \frac{1}{t} \\
         &= \mathcal{LN}( t | x'\beta, \sigma^2 )
\end{align*}
where $\mathcal{N}(\cdot|\mu, \sigma^2)$ is the normal density of mean $\mu$ and $\sigma^2$, and
$\mathcal{LN}(\cdot|\mu, \sigma^2)$ is the corresponding log-normal density.

Also
\begin{align*}
    \log T  &= x'\beta + \sigma \varepsilon \\
    \varepsilon &\sim \mathcal{N}(0, 1) \\
\end{align*}
Or
\begin{align*}
     T  &= \exp(x'\beta) v \\
     v  &= \exp(\sigma \varepsilon) \\
     \varepsilon &\sim \mathcal{N}(0, 1)
\end{align*}

If data is censored, likelihood may be calculated by integrating $f$ over ranges where the true $t$ may lie.

\section{Mixtures of Experts of AFTs}

Idea: Mixture of log-normal AFTs where weight of each mixture components depends on
the covariates $x$; each AFT component becomes an ``expert'' for certain region of the input space.

Let $\theta = \left\{ \beta_{1:J}, \rho_{1:J}, \sigma^2_{1:J} \right\}$.
\begin{align*}
    f(t|x, \theta) = \sum_{j}^J p_j(x, \rho_{1:J}) f_j(t|x, \beta_j, \sigma^2_j)
\end{align*}
Where $f_j$ are log-normal AFT. We use a logistic link
\begin{align*}
    p_j(x, \rho_{1:J}) &= \frac{\exp(x'\rho_j)}{\sum_l^J \exp(x'\rho_l)}
\end{align*}
Also
\begin{align*}
    F(t|x, \theta) = \sum_{j}^J p_j(x, \rho_{1:J}) F_j(t|x, \beta_j, \sigma^2_j)
\end{align*}
\section{Likelihood}
We asume that failture times are  independently distributed and independent from the censoring process. For each individual we observe the failure time, a right censored or interval censored failure time as usually define.

The likelihood is then (Cox and Oakes, 1984)
\begin{align}
\nonumber    L(\mathbf \theta | \mathbf{t} , \mathbf{X}) = \prod_{i\in \mathcal{U}}{} f(t_i| x_i, \mathbf{\theta})\prod_{i\in \mathcal{C} }\{1-F(t_i^+|x_i, \mathbf{\theta} \}\\
    \times \prod_{i\in \mathcal{I} }\{F(t_{iU}|x_i, \mathbf{\theta})-F(t_{iL}|x_i, \mathbf{\theta}) \}
\end{align}

\section{Inference in the Mixtures of AFT Model Experts}
We use MCMC with a data augmentation approach, where ``true'' survival times $
\mathbf{w}$ and component membership $\mathbf{Z}$ (where $Z_{ij}=1$ indicates
observation $i$ comes from component $j$) are considered latent variables.

%If $w$ and $\mathbf{Z}$ are known, likelihood calculation is simple.

Pseudo-observations $w_i$, $i=1,\dots, n$,  are sampled. If $t_i$ is not censored, $w_i = t_i$. If $t_i$ is interval censored, $w_i$ is sampled by solving
\begin{eqnarray}\label{eqwi}
    w_i = F^{-1}\left( F(t_{iL} | x_i, \theta ) + u\left( F(t_{iU}|x_i, \theta) - F(t_{iL}|x_i, \theta) \right)  \right)
\end{eqnarray}
where $t_{iU}$ and $t_{iL}$ are the upper and lower bounds for the time of occurrence of the event, and $u$ is a uniform(0,1) random sample.
If $t_i$ is right censored, $w_i$ is sampled by solving
\begin{eqnarray}\label{eqwi2}
    w_i = F^{-1}\left( F(t_i^+ | x_i, \theta ) + u\left( 1 - F(t_i^+|x_i, \theta)\right)  \right)
\end{eqnarray}
where $t_{i}^+$ is the lower bound for the occurrence of $t_i$, and $u$ is a uniform(0,1) random sample.

$z_i$, $i=1\dots n$ are sampled, where $z_i$ is a sample from a multinomial distribution with parameters $(1, h_{i1}, \dots, h_{iJ})$, i.e.,
\begin{eqnarray}\label{eqzi}
    z_{i} \sim Multinomial (1, h_{i1}, \dots, h_{iJ})
\end{eqnarray}
with
\begin{align*}
    h_{ij} = \frac{ p_j(x_i, \rho_{1:J}) f_j(w_i | x_i, \beta_j) } { \sum_l^J p_l(x_i, \rho_{1:J}) f_l(w_i | x_i, \beta_l) }
\end{align*}
then the augmented likelihood for the completely imputed data $\mathbf w$ and $\mathbf z$ is
\begin{align*}
    L(\theta | \mathbf z, \mathbf w, \mathbf t ,\mathbf X) = \prod_{i=1}^n \prod_{j=1}^J (p_j(\mathbf x_i,\rho ) f_j(w_i| x_i,\beta_j , \sigma^2_j))^{z_{ij}}
\end{align*}
We consider as prior distributions $(\beta_j | \sigma^2_j) \sim \mathcal{N} (b_{0j},\sigma^2_j B_{0j})$, and $\sigma^2_j \sim \mathcal{IG}(n_{0j}/2,n_{0j}S_{0j}/2)$ independent for all $j=1,\ldots , J$. And independently for $\rho \sim \mathcal{N}(\mathbf{0},1000 \times I)$. Then, the posterior distribution for the parameters is given by
\begin{eqnarray*}
    \pi(\theta | \mathbf z, \mathbf w, \mathbf t ,\mathbf X) &=& \prod_{i=1}^n \prod_{j=1}^J (p_j(\mathbf x_i,\rho ) f_j(w_i| x_i,\beta_j , \sigma^2_j))^{z_{ij}} \{\prod_{j=1}^J \phi (\beta_j | \mathbf b_{0j},\sigma^2_j \mathbf B_{0j}) \times \mathcal{IG} (\sigma^2_j| n_{0j}/2,n_{0j}S_{0j}/2 )\} \\
    & & ~~~~~~~~~~~~~~~~~~~~~~~~~~~~~~~~~~~~~~~~~~~~~~~\times \phi (\rho | \mathbf 0,1000\times I)
\end{eqnarray*}

Algorithm:
\begin{enumerate}
    \item Choose $\mathbf{w^0}$, $\mathbf{Z}^0$, $\theta^0 = \left\{ \sigma^{2(0)}_{1:J}, \beta_{1:J}^0, \rho_{1:J}^0 \right\}$
    \item Sample $f( \mathbf{w} | \theta, \mathbf{Z} )$ as equations \eqref{eqwi} - \eqref{eqwi2}
    \item Sample $f( \mathbf{Z} | \theta, \mathbf{w} )$ as equation \eqref{eqzi}
    \item Sample $f( \theta | \mathbf{Z}, \mathbf{w})$, is explained next
    \item Repeat 2-4 until convergence
\end{enumerate}

Step 4, at this step we decide to blocking parameter $\mathbf \theta$ into $\mathbf \beta$, $\mathbf \sigma$ and $\mathbf \delta$, as follows

Note that, $\mathbf z = (z_1,\ldots , z_n)$ defines the group (or cluster, or component) from which the observation becomes.
% Note que z define de que grupo la observación proviene

Thus on each group (or cluster), the regression parameters $\mathbf \beta_j$, $j=1,\ldots,J$, are define by the sample $(\mathbf Y_j ,\mathbf X_j)$, i.e. the observations that becomes from the $j$ group (or cluster), then the full conditional posterior distribution for $\mathbf \beta_j$ is given by
\begin{eqnarray*}
    \mathbf{ \beta}_j | \ldots  &\sim &\mathcal{N}( \mathbf b_{1j}, \sigma^2_j \mathbf B_{1j})\\
    \mathbf \sigma^2_j | \ldots & \sim &\mathcal{IG}( n_1 / 2 , n_1S_1/2)
\end{eqnarray*}
where $\mathcal{IG}$ denotes the inverse gamma distribution.
\begin{eqnarray*}
\mathbf b_{1j} &=& \mathbf B_{1j} ( \mathbf B^{-1}_{0j}b_{0j} + \mathbf X_{j}' \mathbf Y_{j} )\\
\mathbf B_{1j}^{-1} &=& \mathbf B_{0j}^{-1} + \mathbf X_{j}'  \mathbf X_{j}\\
n_1 &=& n_{0j} + n_j\\
n_1 S_1 &=& n_{0j} S_{0j} + (n_j - p ) S_j^2 + (\hat{\beta}_j - \mathbf b_{0j})' (\mathbf B_{0j} + (\mathbf X_{j}'  \mathbf X_{j})^{-1})^{-1} (\hat{\beta}_j - \mathbf b_{0j})
\end{eqnarray*}
and $\hat{\beta}_j = (\mathbf X_{j}' \mathbf X_{j})^{-1} \mathbf X_{j}' \mathbf Y_{j} $, $S_j^2 = \mathbf Y'_j(\mathbf I_n - \mathbf X_j (\mathbf X'_j \mathbf X_j)^{-1} \mathbf X'_j)/(n_j - p)$.

Note that, if $n_0\rightarrow 0$ and $B_0^{-1} \rightarrow 0$ then the priori is non-informative.

For the $\mathbf \rho$ parameters a random walk Metropolis-Hastings step is used, due to the non-tractability of the full conditional posterior for this block of parameters.

Observe that, the full conditional posterior distribution for $\rho_j$ given the rest of parameters and observations is given by
\begin{eqnarray*}
\pi (\rho_j| \mathbf \rho_{-j} , \ldots ) &\propto & \prod_{l \in j} \frac{exp(\mathbf x_l \mathbf \rho_j)}{\sum_{j =1}^J exp(\mathbf x_l \mathbf \rho_j)}
\end{eqnarray*}

\end{document}
